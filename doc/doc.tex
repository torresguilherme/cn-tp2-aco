\documentclass[11pt]{article}
\usepackage[utf8]{inputenc}
\usepackage[T1]{fontenc}
\usepackage{amssymb}
\usepackage{booktabs}
\usepackage{array}
\usepackage{filecontents}
\usepackage{pgfplotstable}
\pgfplotstableset{
  empty cells with={---},
  every head row/.style={before row=\toprule,after row=\midrule},
  every last row/.style={after row=\bottomrule}
}
\pgfplotsset{compat=1.9}

\begin{document}
\title{Computação Natural - TP2\\Ant Colony Optimization}
\author{Guilherme Torres\\Departamento de Ciência da Computação - UFMG}
\date{}
\maketitle

\section{Introdução}

\section{Implementação}

\subsection{Fundamentação}

O funcionamento de uma otimização de colônia de formigas, modelada especificamente para esse problema, pode ser descrito, no plano geral, pelos seguintes passos:

\begin{enumerate}
	\item Inicializa as formigas e uma matriz de feromônios n x n, com o valor 0.5 em cada posição, mais um vetor de feromônios de tamanho n para escolher as medianas;
	\item Distribui as formigas entre exatamente p medianas, escolhendo a probabilidade de acordo com o feromônio no vetor;
	\item Para cada formiga, escolha um cliente para a mediana que ela está, em função da matriz de feromônios;
	\item Encontre a fitness bruta da solução, calculando o somatório da distância que todas as formigas andaram;
	\item Caso haja um excedente à capacidade de uma mediana, aplique uma penalidade (capacidade excedida $* P$, sendo $P$ um modificador de penalidade) à fitness;
	\item Deixe no vetor e matriz de feromônios feedback positivo ou negativo, dependendo se a solução encontrada foi melhor do que a melhor obtida até o momento ou não;
	\item Imprima a média das fitness e a melhor encontrada até o momento;
	\item Repita desde o passo 2, até que o número $N$ de iterações seja atingido;
	\item Termine a execução com a melhor fitness encontrada.
\end{enumerate}

\subsection{Decisões de implementação}

\section{Experimentos}

\section{Conclusões}

\section{Referências}

\end{document}